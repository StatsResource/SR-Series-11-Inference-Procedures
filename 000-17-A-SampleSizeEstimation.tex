
\chapter{15. Sample Size Estimation}

\textbf{Margin of Error}

\begin{itemize}
\item  The product of the quantile and the standard error give us the width of the confidence interval
\item  The width of the confidence interval is known as the \textbf{\emph{margin of error}}.  \[ \mbox{Margin of Error}  = \left[ \mbox{Quantile} \times \mbox{Standard Error} \right] \]
\item  The margin of error gives us some idea about how uncertain we are about the unknown population parameter. \item  A very wide interval may indicate that more data should be collected before anything very definite can be said about the parameter.
\item  The only way to control the margin of error is to adjust the sample size accordingly.
\item  By choosing an appropriate sample size, it is possible to ensure that the margin of error does not excess a certain threshold.
\end{itemize}







\section{Margin of Error}


\section{Sample Size Estimation}

\textbf{Margin of Error}

\begin{itemize}
\item The product of the quantile and the standard error give us the width of the confidence interval
\item The width of the confidence interval is known as the \textbf{\emph{margin of error}}.  \[ \mbox{Margin of Error}  = \left[ \mbox{Quantile} \times \mbox{Standard Error} \right] \]
\item The margin of error gives us some idea about how uncertain we are about the unknown population parameter. \item A very wide interval may indicate that more data should be collected before anything very definite can be said about the parameter.
\item The only way to control the margin of error is to adjust the sample size accordingly.
\item By choosing an appropriate sample size, it is possible to ensure that the margin of error does not excess a certain threshold.
\end{itemize}
\newpage

\textbf{Margin of Error}

\begin{itemize}
\item The product of the quantile and the standard error give us the width of the confidence interval
\item The width of the confidence interval is known as the \textbf{\emph{margin of error}}.  \[ \mbox{Margin of Error}  = \left[ \mbox{Quantile} \times \mbox{Standard Error} \right] \]
\item The margin of error gives us some idea about how uncertain we are about the unknown population parameter. \item A very wide interval may indicate that more data should be collected before anything very definite can be said about the parameter.
\item The only way to control the margin of error is to adjust the sample size accordingly.
\item By choosing an appropriate sample size, it is possible to ensure that the margin of error does not excess a certain threshold.
\end{itemize}



%------------------------------------------------------------------------------------------------%
\section{Sample Size Estimation}

The larger your sample size, the more sure you can be that their answers truly reflect the population.\\ This indicates that for a given confidence level, the larger your sample size, the smaller your confidence interval. \\However, the relationship is not linear (i.e., doubling the sample size does not halve the confidence interval).

%------------------------------------------------------------------------------------------------%
\section{Sample size Estimation}
For a certain variable, the standard deviation in a large population is equal to 12.5.
How big a sample is needed to be 95\% sure that the sample mean is within 1.5 units of the population mean?


\begin{itemize}
\item For a certain variable, the standard deviation in a large population is equal to 8.5.
\item How big a sample is needed to be 90\% sure that the sample mean is within 1.5
units of the population mean?
\end{itemize}
%------------------------------------------------------------------------------------------------%




\subsection{SSE for the Mean: Example}

\begin{itemize}
\item An IT training company has developed a new certification program. The company wishes to estimate the average score of those who complete the program by self-study.  \item The standard deviation of the self study group is assumed to be the same as the overall population of candidates, ie. 21.2 points.
\item How many people must be tested if the sample mean is to be in error by no more than 3 points, with 95\% confidence.
\end{itemize}



%--------------------------------------------------------%

% \subsection{SSE for the Mean: Example}

\begin{itemize}
\item The sample size we require is the smallest value for $n$ which satisfies this identity.
\[ n \geq \frac{\sigma^2 Q^2_{(1-\alpha)}}{E^2}  \]
\item Remark: $1-\alpha$ = 0.95, therefore $Q_{(1-\alpha)}$ = 1.96. Also $E=3$ and $\sigma =21.2$.
\[ n \geq \frac{(21.2)^2 \times (1.96)^2}{3^2} \]
\item Solving, the required sample size is the smallest value of $n$ that satisfies
\[ n \geq 191.8410 \]
\item Therefore, the company needs to test 192 self-study candidates.
\end{itemize}





\section{Example}
\begin{itemize}
\item An accounting firm wishes to test the claim that no more than 1\% of a large
number of transactions contains errors. In order to test this claim, they
examine a random sample of 144 transactions and find that exactly 3 of
these are in error.

\item An accounting firm wishes to test the claim that no more than 5\% of a large
number of transactions contains errors. In order to test this claim, they examine a
random sample of 225 transactions and find that exactly 20 of these are in error.
\end{itemize}








%--------------------------------------------------------%

\noindent \textbf{Sample Size Estimation for proportions}
We can also compute appropriate sample sizes for studies based on proportions.
\begin{itemize}
\item From before; \[ E \geq Q \times S.E.(\hat{p}). \]
(For the sake of brevity, we will just use the notation $Q$ for quantile.)

\item Divide both sides by Q.

\[ E \geq Q \times \sqrt{ {\pi(1-\pi)  \over n} }. \]

\end{itemize}


\begin{itemize}
\item Remark: $E$ must be expressed in the same form as $\pi$, either as a proportion or as a percentage.
\item Remark : The standard error is maximized at $\pi = 0.50$,which is to say $\pi(1-\pi)$ can never exceed 0.25 ( or 25\%). Therefore the standard error is maximized at $\pi = 0.50$. To make the procedure as conservative as possible, we will use $0.25$ as our value for $\hat{p}_1 \times (1 - \hat{p}_1)$.
\item If we use percentages, $\pi \times (100-\pi)$ can not exceed 2500 (i.e $ 50 \times (100-50)=2500)$.

\[ E \geq Q \times \sqrt{{2500 \over n}}. \]


\end{itemize}


%--------------------------------------------------------%

\noindent \textbf{Sample Size Estimation for Proportions}

\begin{itemize}

\item Dive both sides by $Q$, the square both sides:

\[ \left({E\over Q}\right)^2 \geq {2500 \over n}. \]

\item Invert both sides, changing the direction of the relational operator, and multiply both sides by $2500$.

\[ \left({Q\over E}\right)^2 \times 2500 \leq n. \]

\item The sample size we require is the smallest value for $n$ which satisfies this identity. (This formula would be provided on the exam paper, but without the maximized standard error).
\end{itemize}





%--------------------------------------------------------%

\newpage





%--------------------------------------------------------%

\textbf{Sample Size Estimation for proportions}
\begin{itemize}
\item Remark: $E$ must be expressed in the same form as $\pi$, either as a proportion or as a percentage.
\item Remark : The standard error is maximized at $\pi = 0.50$,which is to say $\pi(1-\pi)$ can never exceed 0.25 ( or 25\%). Therefore the standard error is maximized at $\pi = 0.50$. To make the procedure as conservative as possible, we will use $0.25$ as our value for $\hat{p}_1 \times (1 - \hat{p}_1)$.
\item If we use percentages, $\pi \times (100-\pi)$ can not exceed 2500 (i.e $ 50 \times (100-50)=2500)$.

\[ E \geq Q \times \sqrt{{2500 \over n}}. \]


\end{itemize}


%--------------------------------------------------------%

\textbf{Sample Size Estimation for proportions}

\begin{itemize}

\item Dive both sides by $Q$, the square both sides:

\[ \left({E\over Q}\right)^2 \geq {2500 \over n}. \]

\item Invert both sides, changing the direction of the relational operator, and multiply both sides by $2500$.

\[ \left({Q\over E}\right)^2 \times 2500 \leq n. \]

\item The sample size we require is the smallest value for $n$ which satisfies this identity. (This formula would be provided on the exam paper, but without the maximized standard error).
\end{itemize}




%----------------------------------------------------------------------------------------------------%
{
\begin{itemize}
\item $\mu_d$ mean value for the population of differences.
\item $\bar{d}$ mean value for the sample of differences,
\item $s_d$ standard deviation of the differences for the paired sample data.
\item $n$ number of pairs
\end{itemize}


}




%---------------------------------------------------------------------------------------------%
{
\textbf{The Critical region}
The critical region ( or rejection region ) is the set of all values of the test statistic that causes us to rejec the null hypothesis.

}
{

Test statistics for testing a claim about a mean, when the population variance is known.

\[ Z = {\bar{x}  - \mu \over {\sigma \over \sqrt{n}}} \]
}





%--------------------------------------------------------------------------------------------------------------------------%

\textbf{P-values}

\begin{itemize}
\item The null hypothesis either is or is not rejected at the previously stated significance level. Thus, if an experimenter originally stated that he or she was using the $\alpha = 0.05$ significance level and p-value was subsequently calculated to be $0.042$, then the person would reject the null hypothesis at the 0.05 level. \item If p-value had been 0.0001 instead of 0.042 then the null hypothesis would still be rejected at the 0.05 significance level.  \item
The experimenter would not have any basis to be more confident that the null hypothesis was false with a p-value of 0.0001 than with a p-value of 0.041. \item Similarly, if the p had been 0.051 then the experimenter would fail to reject the null hypothesis
\end{itemize}











%--------------------------------------------------------%

\textbf{Sample Size Estimation for proportions}

\begin{itemize}

\item Dive both sides by $Q$, the square both sides:

\[ \left({E\over Q}\right)^2 \geq {2500 \over n}. \]

\item Invert both sides, changing the direction of the relational operator, and multiply both sides by $2500$.

\[ \left({Q\over E}\right)^2 \times 2500 \leq n. \]

\item The sample size we require is the smallest value for $n$ which satisfies this identity. (The formula,  depicted two slides previously, would be provided on the exam paper, but without the maximized standard error).
\end{itemize}



%-------------------------------------------------------%

\textbf{Sample Size Estimation}

\begin{itemize} \item Recall:
\[  \mbox{Margin of Error}  = \mbox{Quantile} \times \mbox{Std. Error}\]

\item Also recall that the only way to influence the margin of error is to set the sample size accordingly.

\item Sample size estimation describes the selection of a sample size such that the margin does not exceed a preddetermined level.
\end{itemize}


%--------------------------------------------------------%

\textbf{Sample Size estimation for the mean}

\begin{itemize}
\item \[ E \geq Q \times S.E.(\bar{x}). \]

\item 
$E \geq Q \times {\sigma \over \sqrt{n} }$

\item
\[ \frac{E}{\sigma Q} \geq {1 \over \sqrt{n} } \]

\item Square both sides


\[ \frac{E^2}{\sigma^2 Q^2} \geq {1 \over n } \]


\end{itemize}


%--------------------------------------------------------%

\textbf{Sample Size estimation for proportions}

\begin{itemize}
\item \[ E \geq Q \times S.E.(\hat{p}). \]

\item 
\[ E \geq Q \times \sqrt{{\hat{p}(1-\hat{p} \over n}}. \]

\item Remark : $\hat{p} \times (1-\hat{p})$

\item Square both sides



\end{itemize}



%--------------------------------------------------------%

\textbf{Sample Size estimation for proportions}

$\left[1.96 \times \sqrt{{ 50 \times 50 \over n}} \right]< 4 $


$\left[1.96 \times \sqrt{{ 2500 \over n}} \right]< 4 $

$\left[ \sqrt{{ 2500 \over n}} \right]< {4 \over 1.96}$

$\left[ { 2500 \over n} \right]< {4^2 \over 1.96^2}$

$\left[ { n \over 2500} \right]> {1.96^2 \over 4^2}$

$n> {1.96^2 \over 4^2} \times 2500$

$n>600.25$ 
n=601

%--------------------------------------------------------%


\textbf{Sample Size estimation for proportions}


$n> {Q^2E^2 \over (0.25)}$
$n >{Q^2 E^2 \over (2500)}$



%-------------------------------------------------------%

\textbf{Sample Size Estimation}

\begin{itemize} \item Recall the formula for margin of error, which we shall denote $E$.
\[  E = Q_{(1-\alpha)} \times \mbox{Std. Error}\]

\item $Q_{(1-\alpha)}$ denotes the quantile that corresponds to a $1-\alpha$ confidence level. (There is quite a bit of variation in notation in this respect.)
\item Also recall that the only way to influence the margin of error is to set the sample size accordingly.

\item Sample size estimation describes the selection of a sample size $n$ such that the margin of error does not exceed a pre-determined level $E$.
\end{itemize}


%--------------------------------------------------------%

\textbf{Sample Size Estimation for the Mean}

\begin{itemize}

\item The margin of error does not exceed a certain threshold $E$.
\[ E \geq Q_{(1-\alpha)} \times S.E.(\bar{x}), \]

\item which can be re-expressed as
\[E \geq Q_{(1-\alpha)} \times {\sigma \over \sqrt{n} }.\]

\item Divide both sides by $\sigma \times Q_{(1-\alpha)}$.
\[ \frac{E}{\sigma Q_{(1-\alpha)}} \geq {1 \over \sqrt{n} } \]

\item Square both sides

\[ \frac{E^2}{\sigma^2 Q_{(1-\alpha)}^2} \geq {1 \over n } \]


\end{itemize}

%--------------------------------------------------------%

\textbf{Sample Size Estimation for the Mean}

\begin{itemize}
\item Square both sides

\[ \frac{E^2}{\sigma^2 Q^2_{(1-\alpha)}} \geq {1 \over n } \]

\item Invert both sides, changing the direction of the relational operator.

\[ \frac{\sigma^2 Q^2_{(1-\alpha)}}{E^2} \leq n \]


\item The sample size we require is the smallest value for $n$ which satisfies this identity.
\item The sample standard deviation $s$ may be used as an estimate for $\sigma$.
\item (This formula would be provided on the exam paper).
\end{itemize}




%--------------------------------------------------------%

\textbf{SSE for the Mean: Example}




%--------------------------------------------------------%

\textbf{Sample Size Estimation for proportions}
We can also compute appropriate sample sizes for studies based on proportions.
\begin{itemize}
\item From before; \[ E \geq Q \times S.E.(\hat{p}). \]
(For the sake of brevity, we will just use the notation $Q$ for quantile.)

\item Divide both sides by Q.

\[ E \geq Q \times \sqrt{ {\pi(1-\pi)  \over n} }. \]

\end{itemize}


%--------------------------------------------------------%

\textbf{Sample Size Estimation for proportions}
\begin{itemize}
\item Remark: $E$ must be expressed in the same form as $\pi$, either as a proportion or as a percentage.
\item Remark : The standard error is maximized at $\pi = 0.50$,which is to say $\pi(1-\pi)$ can never exceed 0.25 ( or 25\%). Therefore the standard error is maximized at $\pi = 0.50$. To make the procedure as conservative as possible, we will use $0.25$ as our value for $\hat{p}_1 \times (1 - \hat{p}_1)$.
\item If we use percentages, $\pi \times (100-\pi)$ can not exceed 2500 (i.e $ 50 \times (100-50)=2500)$.

\[ E \geq Q \times \sqrt{{2500 \over n}}. \]


\end{itemize}


%--------------------------------------------------------%

\textbf{Sample Size Estimation for proportions}

\begin{itemize}

\item Dive both sides by $Q$, the square both sides:

\[ \left({E\over Q}\right)^2 \geq {2500 \over n}. \]

\item Invert both sides, changing the direction of the relational operator, and multiply both sides by $2500$.

\[ \left({Q\over E}\right)^2 \times 2500 \leq n. \]

\item The sample size we require is the smallest value for $n$ which satisfies this identity. (This formula would be provided on the exam paper, but without the maximized standard error).
\end{itemize}

%--------------------------------------------------------%

\textbf{SSE for proportions: Example}
\begin{itemize}
\item An IT journal wants to conduct a survey to estimate the true proportion of university students that own laptops.
\item The journal has decided to uses a confidence level of $95\%$, with a margin of error of $2\%$.
\item How many university students must be surveyed?
\end{itemize}


%--------------------------------------------------------%

\textbf{Sample Size estimation for proportions}

\begin{itemize}
\item Confidence level = 0.95. Therefore the quantile is $Q_{(1-\alpha)} = 1.96$
\item Using the formula: \[ n \geq \left({1.96 \over 2 }\right)^2 \times 2500  \]
\item The required sample size is the smallest value for $n$ which satisfies this identity: \[ n \geq 2401  \]
\item The required sample size is therefore 2401.
\end{itemize}









\textbf{Sample Size Estimation for proportions}

\begin{itemize}

\item Dive both sides by $Q$, the square both sides:

\[ \left({E\over Q}\right)^2 \geq {2500 \over n}. \]

\item Invert both sides, changing the direction of the relational operator, and multiply both sides by $2500$.

\[ \left({Q\over E}\right)^2 \times 2500 \leq n. \]

\item The sample size we require is the smallest value for $n$ which satisfies this identity. (The formula,  depicted two slides previously, would be provided on the exam paper, but without the maximized standard error).
\end{itemize}







%--------------------------------------------------------%

\textbf{Sample Size Estimation for proportions}
We can also compute appropriate sample sizes for studies based on proportions.
\begin{itemize}
\item From before; \[ E \geq Q \times S.E.(\hat{p}). \]
(For the sake of brevity, we will just use the notation $Q$ for quantile.)

\item Divide both sides by Q.

\[ E \geq Q \times \sqrt{ {\pi(1-\pi)  \over n} }. \]

\end{itemize}


%--------------------------------------------------------%

\textbf{Sample Size Estimation for proportions}
\begin{itemize}
\item Remark: $E$ must be expressed in the same form as $\pi$, either as a proportion or as a percentage.
\item Remark : The standard error is maximized at $\pi = 0.50$,which is to say $\pi(1-\pi)$ can never exceed 0.25 ( or 25\%). Therefore the standard error is maximized at $\pi = 0.50$. To make the procedure as conservative as possible, we will use $0.25$ as our value for $\hat{p}_1 \times (1 - \hat{p}_1)$.
\item If we use percentages, $\pi \times (100-\pi)$ can not exceed 2500 (i.e $ 50 \times (100-50)=2500)$.

\[ E \geq Q \times \sqrt{{2500 \over n}}. \]


\end{itemize}


%--------------------------------------------------------%

\textbf{Sample Size Estimation for proportions}

\begin{itemize}

\item Dive both sides by $Q$, the square both sides:

\[ \left({E\over Q}\right)^2 \geq {2500 \over n}. \]

\item Invert both sides, changing the direction of the relational operator, and multiply both sides by $2500$.

\[ \left({Q\over E}\right)^2 \times 2500 \leq n. \]

\item The sample size we require is the smallest value for $n$ which satisfies this identity. (This formula would be provided on the exam paper, but without the maximized standard error).
\end{itemize}

%--------------------------------------------------------%

\textbf{SSE for proportions: Example}
\begin{itemize}
\item An IT journal wants to conduct a survey to estimate the true proportion of university students that own laptops.
\item The journal has decided to uses a confidence level of $95\%$, with a margin of error of $2\%$.
\item How many university students must be surveyed?
\end{itemize}


%--------------------------------------------------------%

\textbf{Sample Size estimation for proportions}

\begin{itemize}
\item Confidence level = 0.95. Therefore the quantile is $Q_{(1-\alpha)} = 1.96$
\item Using the formula: \[ n \geq \left({1.96 \over 2 }\right)^2 \times 2500  \]
\item The required sample size is the smallest value for $n$ which satisfies this identity: \[ n \geq 2401  \]
\item The required sample size is therefore 2401.
\end{itemize}

\section{Sample Size Estimation}

%-------------------------------------------------------%

\textbf{Sample Size Estimation}
\textbf{New Section} This is the started of the second section of inference procedures.
\begin{itemize} \item  Recall the formula for margin of error, which we shall denote $E$.
\[  E = Q_{(1-\alpha)} \times \mbox{Std. Error}\]

\item  $Q_{(1-\alpha)}$ denotes the quantile that corresponds to a $1-\alpha$ confidence level. (There is quite a bit of variation in notation in this respect.)
\item  Also recall that the only way to influence the margin of error is to set the sample size accordingly.

\item  Sample size estimation describes the selection of a sample size $n$ such that the margin of error does not exceed a pre-determined level $E$.
\end{itemize}


%--------------------------------------------------------%

\textbf{Sample Size Estimation for the Mean}

\begin{itemize}

\item  The margin of error does not exceed a certain threshold $E$.
\[ E \geq Q_{(1-\alpha)} \times S.E.(\bar{x}), \]

\item  which can be re-expressed as
\[E \geq Q_{(1-\alpha)} \times {\sigma \over \sqrt{n} }.\]

\item  Divide both sides by $\sigma \times Q_{(1-\alpha)}$.
\[ \frac{E}{\sigma Q_{(1-\alpha)}} \geq {1 \over \sqrt{n} } \]

\item  Square both sides

\[ \frac{E^2}{\sigma^2 Q_{(1-\alpha)}^2} \geq {1 \over n } \]

\end{itemize}

%--------------------------------------------------------%

\textbf{Sample Size Estimation for the Mean}
\begin{itemize}
\item  Square both sides
\[ \frac{E^2}{\sigma^2 Q^2_{(1-\alpha)}} \geq {1 \over n } \]
\item  Invert both sides, changing the direction of the relational operator.
\[ \frac{\sigma^2 Q^2_{(1-\alpha)}}{E^2} \leq n \]

\item  The sample size we require is the smallest value for $n$ which satisfies this identity.
\item  The sample standard deviation $s$ may be used as an estimate for $\sigma$.
\item  (This formula would be provided on the exam paper).
\end{itemize}


%--------------------------------------------------------%

\textbf{SSE for the Mean: Example}
\begin{itemize}
\item  An IT training company has developed a new certification program. The company wishes to estimate the average score of those who complete the program by self-study.  \item  The standard deviation of the self study group is assumed to be the same as the overall population of candidates, ie. 21.2 points.
\item  How many people must be tested if the sample mean is to be in error by no more than 3 points, with 95\% confidence.
\end{itemize}


\textbf{Sample Size Estimation}
\textbf{New Section} This is the started of the second section of inference procedures.
\begin{itemize} \item  Recall the formula for margin of error, which we shall denote $E$.
\[  E = Q_{(1-\alpha)} \times \mbox{Std. Error}\]

\item  $Q_{(1-\alpha)}$ denotes the quantile that corresponds to a $1-\alpha$ confidence level. (There is quite a bit of variation in notation in this respect.)
\item  Also recall that the only way to influence the margin of error is to set the sample size accordingly.

\item  Sample size estimation describes the selection of a sample size $n$ such that the margin of error does not exceed a pre-determined level $E$.
\end{itemize}


%--------------------------------------------------------%

\textbf{Sample Size Estimation for the Mean}

\begin{itemize}

\item  The margin of error does not exceed a certain threshold $E$.
\[ E \geq Q_{(1-\alpha)} \times S.E.(\bar{x}), \]

\item  which can be re-expressed as
\[E \geq Q_{(1-\alpha)} \times {\sigma \over \sqrt{n} }.\]

\item  Divide both sides by $\sigma \times Q_{(1-\alpha)}$.
\[ \frac{E}{\sigma Q_{(1-\alpha)}} \geq {1 \over \sqrt{n} } \]

\item  Square both sides

\[ \frac{E^2}{\sigma^2 Q_{(1-\alpha)}^2} \geq {1 \over n } \]

\end{itemize}

%--------------------------------------------------------%

\textbf{Sample Size Estimation for the Mean}
\begin{itemize}
\item  Square both sides
\[ \frac{E^2}{\sigma^2 Q^2_{(1-\alpha)}} \geq {1 \over n } \]
\item  Invert both sides, changing the direction of the relational operator.
\[ \frac{\sigma^2 Q^2_{(1-\alpha)}}{E^2} \leq n \]

\item  The sample size we require is the smallest value for $n$ which satisfies this identity.
\item  The sample standard deviation $s$ may be used as an estimate for $\sigma$.
\item  (This formula would be provided on the exam paper).
\end{itemize}


%--------------------------------------------------------%

\textbf{SSE for the Mean: Example}
\begin{itemize}
\item  An IT training company has developed a new certification program. The company wishes to estimate the average score of those who complete the program by self-study.  \item  The standard deviation of the self study group is assumed to be the same as the overall population of candidates, ie. 21.2 points.
\item  How many people must be tested if the sample mean is to be in error by no more than 3 points, with 95\% confidence.
\end{itemize}

%--------------------------------------------------------%

\textbf{SSE for the Mean: Example}

\begin{itemize}
\item  The sample size we require is the smallest value for $n$ which satisfies this identity.
\[ n \geq \frac{\sigma^2 Q^2_{(1-\alpha)}}{E^2}  \]
\item  Remark: $1-\alpha$ = 0.95, therefore $Q_{(1-\alpha)}$ = 1.96. Also $E=3$ and $\sigma =21.2$.
\[ n \geq \frac{(21.2)^2 \times (1.96)^2}{3^2} \]
\item  Solving, the required sample size is the smallest value of $n$ that satisfies
\[ n \geq 191.8410 \]
\item  Therefore, the company needs to test 192 self-study candidates.
\end{itemize}


%--------------------------------------------------------%

\textbf{Sample Size Estimation for Proportions}
We can also compute appropriate sample sizes for studies based on proportions.
\begin{itemize}
\item  From before; \[ E \geq Q \times S.E.(\hat{p}). \]
(For the sake of brevity, we will just use the notation $Q$ for quantile.)

\item  Divide both sides by Q.

\[ E \geq Q \times \sqrt{ {\pi(1-\pi)  \over n} }. \]

\end{itemize}


%--------------------------------------------------------%

\textbf{Sample Size Estimation for Proportions}
\begin{itemize}
\item  Remark: $E$ must be expressed in the same form as $\pi$, either as a proportion or as a percentage.
\item  Remark : The standard error is maximized at $\pi = 0.50$,which is to say $\pi(1-\pi)$ can never exceed 0.25 ( or 25\%). Therefore the standard error is maximized at $\pi = 0.50$. To make the procedure as conservative as possible, we will always use $0.25$ as our value for $\hat{p}_1 \times (1 - \hat{p}_1)$. (Equivalently 2500 for percentages).
\item  If we use percentages, $\pi \times (100-\pi)$ can not exceed 2500 (i.e $ 50 \times (100-50)=2500)$.

\[ E \geq Q \times \sqrt{{2500 \over n}}. \]


\end{itemize}


%--------------------------------------------------------%

\textbf{Sample Size Estimation for proportions}

\begin{itemize}

\item  Dive both sides by $Q$, the square both sides:

\[ \left({E\over Q}\right)^2 \geq {2500 \over n}. \]

\item  Invert both sides, changing the direction of the relational operator, and multiply both sides by $2500$.

\[ \left({Q\over E}\right)^2 \times 2500 \leq n. \]

\item  The sample size we require is the smallest value for $n$ which satisfies this identity. (The formula,  depicted two slides previously, would be provided on the exam paper, but without the maximized standard error).
\end{itemize}

%--------------------------------------------------------%

\textbf{SSE for proportions: Example}
\begin{itemize}
\item  An IT journal wants to conduct a survey to estimate the true proportion of university students that own laptops.
\item  The journal has decided to uses a confidence level of $95\%$, with a margin of error of $2\%$.
\item  How many university students must be surveyed?
\end{itemize}


%--------------------------------------------------------%

\textbf{Sample Size estimation for proportions}

\begin{itemize}
\item  Confidence level = 0.95. Therefore the quantile is $Q_{(1-\alpha)} = 1.96$
\item  Using the formula: \[ n \geq \left({1.96 \over 2 }\right)^2 \times 2500  \]
\item  The required sample size is the smallest value for $n$ which satisfies this identity: \[ n \geq 2401  \]
\item  The required sample size is therefore 2401.
\end{itemize}

%-----------------------------------------------------------%


\subsubsection{Margin of Error}

\begin{itemize}
\item The product of the quantile and the standard error give us the width of the confidence interval
\item The width of the confidence interval is known as the \textbf{\emph{margin of error}}.  \[ \mbox{Margin of Error}  = \left[ \mbox{Quantile} \times \mbox{Standard Error} \right] \]
\item The margin of error gives us some idea about how uncertain we are about the unknown population parameter. \item A very wide interval may indicate that more data should be collected before anything very definite can be said about the parameter.
\item The only way to control the margin of error is to adjust the sample size accordingly.
\item By choosing an appropriate sample size, it is possible to ensure that the margin of error does not excess a certain threshold.
\end{itemize}



%-------------------------------------------------------%

\textbf{Sample Size Estimation}

\begin{itemize} \item  Recall the formula for margin of error, which we shall denote $E$.
\[  E = Q_{(1-\alpha)} \times \mbox{Std. Error}\]

\item  $Q_{(1-\alpha)}$ denotes the quantile that corresponds to a $1-\alpha$ confidence level. (There is quite a bit of variation in notation in this respect.)
\item  Also recall that the only way to influence the margin of error is to set the sample size accordingly.

\item  Sample size estimation describes the selection of a sample size $n$ such that the margin of error does not exceed a pre-determined level $E$.
\end{itemize}


%--------------------------------------------------------%

\textbf{Sample Size Estimation for the Mean}

\begin{itemize}

\item  The margin of error does not exceed a certain threshold $E$.
\[ E \geq Q_{(1-\alpha)} \times S.E.(\bar{x}), \]

\item  which can be re-expressed as
\[E \geq Q_{(1-\alpha)} \times {\sigma \over \sqrt{n} }.\]

\item  Divide both sides by $\sigma \times Q_{(1-\alpha)}$.
\[ \frac{E}{\sigma Q_{(1-\alpha)}} \geq {1 \over \sqrt{n} } \]

\item  Square both sides

\[ \frac{E^2}{\sigma^2 Q_{(1-\alpha)}^2} \geq {1 \over n } \]


\end{itemize}

%--------------------------------------------------------%

\textbf{Sample Size Estimation for the Mean}

\begin{itemize}
\item  Square both sides

\[ \frac{E^2}{\sigma^2 Q^2_{(1-\alpha)}} \geq {1 \over n } \]

\item  Invert both sides, changing the direction of the relational operator.

\[ \frac{\sigma^2 Q^2_{(1-\alpha)}}{E^2} \leq n \]


\item  The sample size we require is the smallest value for $n$ which satisfies this identity.
\item  The sample standard deviation $s$ may be used as an estimate for $\sigma$.
\item  (This formula would be provided on the exam paper).
\end{itemize}







%--------------------------------------------------------%

\textbf{Sample Size Estimation for proportions}
We can also compute appropriate sample sizes for studies based on proportions.
\begin{itemize}
\item  From before; \[ E \geq Q \times S.E.(\hat{p}). \]
(For the sake of brevity, we will just use the notation $Q$ for quantile.)

\item  Divide both sides by Q.

\[ E \geq Q \times \sqrt{ {\pi(1-\pi)  \over n} }. \]

\end{itemize}


%--------------------------------------------------------%

\textbf{Sample Size Estimation for proportions}
\begin{itemize}
\item  Remark: $E$ must be expressed in the same form as $\pi$, either as a proportion or as a percentage.
\item  Remark : The standard error is maximized at $\pi = 0.50$,which is to say $\pi(1-\pi)$ can never exceed 0.25 ( or 25\%). Therefore the standard error is maximized at $\pi = 0.50$. To make the procedure as conservative as possible, we will use $0.25$ as our value for $\hat{p}_1 \times (1 - \hat{p}_1)$.
\item  If we use percentages, $\pi \times (100-\pi)$ can not exceed 2500 (i.e $ 50 \times (100-50)=2500)$.

\[ E \geq Q \times \sqrt{{2500 \over n}}. \]


\end{itemize}


%--------------------------------------------------------%

\textbf{Sample Size Estimation for proportions}

\begin{itemize}

\item  Dive both sides by $Q$, the square both sides:

\[ \left({E\over Q}\right)^2 \geq {2500 \over n}. \]

\item  Invert both sides, changing the direction of the relational operator, and multiply both sides by $2500$.

\[ \left({Q\over E}\right)^2 \times 2500 \leq n. \]

\item  The sample size we require is the smallest value for $n$ which satisfies this identity. (This formula would be provided on the exam paper, but without the maximized standard error).
\end{itemize}

%--------------------------------------------------------%

\textbf{SSE for proportions: Example}
\begin{itemize}
\item  An IT journal wants to conduct a survey to estimate the true proportion of university students that own laptops.
\item  The journal has decided to uses a confidence level of $95\%$, with a margin of error of $2\%$.
\item  How many university students must be surveyed?
\end{itemize}


%--------------------------------------------------------%

\textbf{Sample Size estimation for proportions}

\begin{itemize}
\item  Confidence level = 0.95. Therefore the quantile is $Q_{(1-\alpha)} = 1.96$
\item  Using the formula: \[ n \geq \left({1.96 \over 2 }\right)^2 \times 2500  \]
\item  The required sample size is the smallest value for $n$ which satisfies this identity: \[ n \geq 2401  \]
\item  The required sample size is therefore 2401.
\end{itemize}








\chapter{15. Sample Size Estimation}

%------------------------------------------------------------------------------------------------%
\section{Sample size Estimation}
For a certain variable, the standard deviation in a large population is equal to 12.5.
How big a sample is needed to be 95\% sure that the sample mean is within 1.5 units of the population mean?


\begin{itemize}
\item For a certain variable, the standard deviation in a large population is equal to 8.5.
\item How big a sample is needed to be 90\% sure that the sample mean is within 1.5
units of the population mean?
\end{itemize}
%------------------------------------------------------------------------------------------------%


\subsection{Sample Size Estimation}

\begin{itemize} \item Recall the formula for margin of error, which we shall denote $E$.
\[  E = Q_{(1-\alpha)} \times \mbox{Std. Error}\]

\item $Q_{(1-\alpha)}$ denotes the quantile that corresponds to a $1-\alpha$ confidence level. (There is quite a bit of variation in notation in this respect.)
\item Also recall that the only way to influence the margin of error is to set the sample size accordingly.

\item Sample size estimation describes the selection of a sample size $n$ such that the margin of error does not exceed a pre-determined level $E$.
\end{itemize}


%--------------------------------------------------------%
\section{Sample size Estimation}
For a certain variable, the standard deviation in a large population is equal to 12.5.
How big a sample is needed to be 95\% sure that the sample mean is within 1.5 units of the population mean?


For a certain variable, the standard deviation in a large population is equal to 8.5.
How big a sample is needed to be 90\% sure that the sample mean is within 1.5
units of the population mean?


\subsection{Sample Size Estimation for the Mean}

\begin{multicols}{2}

\begin{itemize}

\item The margin of error does not exceed a certain threshold $E$.
\[ E \geq Q_{(1-\alpha)} \times S.E.(\bar{x}), \]

\item which can be re-expressed as
\[E \geq Q_{(1-\alpha)} \times {\sigma \over \sqrt{n} }.\]

\item Divide both sides by $\sigma \times Q_{(1-\alpha)}$.
\[ \frac{E}{\sigma Q_{(1-\alpha)}} \geq {1 \over \sqrt{n} } \]

\item Square both sides

\[ \frac{E^2}{\sigma^2 Q_{(1-\alpha)}^2} \geq {1 \over n } \]



\item Square both sides

\[ \frac{E^2}{\sigma^2 Q^2_{(1-\alpha)}} \geq {1 \over n } \]


\item Invert both sides, changing the direction of the relational operator.

\[ \frac{\sigma^2 Q^2_{(1-\alpha)}}{E^2} \leq n \]
\end{itemize}

\end{multicols}
\begin{itemize}
\item The sample size we require is the smallest value for $n$ which satisfies this identity.
\item The sample standard deviation $s$ may be used as an estimate for $\sigma$.
\item (This formula would be provided on the exam paper).
\end{itemize}




%--------------------------------------------------------%


\subsection{Sample Size Estimation for proportions}
We can also compute appropriate sample sizes for studies based on proportions.
\begin{itemize}
\item From before; \[ E \geq Q \times S.E.(\hat{p}). \]
(For the sake of brevity, we will just use the notation $Q$ for quantile.)

\item Divide both sides by Q.

\[ E \geq Q \times \sqrt{ {\pi(1-\pi)  \over n} }. \]


\item Remark: $E$ must be expressed in the same form as $\pi$, either as a proportion or as a percentage.
\item Remark : The standard error is maximized at $\pi = 0.50$,which is to say $\pi(1-\pi)$ can never exceed 0.25 ( or 25\%). Therefore the standard error is maximized at $\pi = 0.50$. To make the procedure as conservative as possible, we will use $0.25$ as our value for $\hat{p}_1 \times (1 - \hat{p}_1)$.
\item If we use percentages, $\pi \times (100-\pi)$ can not exceed 2500 (i.e $ 50 \times (100-50)=2500)$.

\[ E \geq Q \times \sqrt{{2500 \over n}}. \]




\item Dive both sides by $Q$, the square both sides:

\[ \left({E\over Q}\right)^2 \geq {2500 \over n}. \]

\item Invert both sides, changing the direction of the relational operator, and multiply both sides by $2500$.

\[ \left({Q\over E}\right)^2 \times 2500 \leq n. \]

\item The sample size we require is the smallest value for $n$ which satisfies this identity. (This formula would be provided on the exam paper, but without the maximized standard error).

\item Confidence level = 0.95. Therefore the quantile is $Q_{(1-\alpha)} = 1.96$
\item Using the formula: \[ n \geq \left({1.96 \over 2 }\right)^2 \times 2500  \]
\item The required sample size is the smallest value for $n$ which satisfies this identity: \[ n \geq 2401  \]
\item The required sample size is therefore 2401.
\end{itemize}




%--------------------------------------------------------%

\subsection{SSE for proportions: Example}
\begin{itemize}
\item An IT journal wants to conduct a survey to estimate the true proportion of university students that own laptops.
\item The journal has decided to uses a confidence level of $95\%$, with a margin of error of $2\%$.
\item How many university students must be surveyed?
\end{itemize}


\section{Example}
\begin{itemize}
\item An accounting firm wishes to test the claim that no more than 1\% of a large
number of transactions contains errors. In order to test this claim, they
examine a random sample of 144 transactions and find that exactly 3 of
these are in error.

\item An accounting firm wishes to test the claim that no more than 5\% of a large
number of transactions contains errors. In order to test this claim, they examine a
random sample of 225 transactions and find that exactly 20 of these are in error.
\end{itemize}




---

# Sample Size Estimation

## General Structure

To control the **margin of error** \( E \) in estimating a population parameter with confidence level \( 1 - \alpha \), we use:

\[
E = Q_{(1-\alpha)} \times \text{Standard Error}
\]

- \( Q_{(1-\alpha)} \) is the quantile associated with the desired confidence level.
- The only way to reduce \( E \) (increase precision) is by increasing the sample size \( n \).

---

## Sample Size Estimation for the Mean

### Margin of Error Constraint

\[
E \geq Q_{(1-\alpha)} \times \frac{\sigma}{\sqrt{n}}
\]

### Rearranging

1. Divide both sides by \( \sigma Q_{(1-\alpha)} \):
\[
\frac{E}{\sigma Q_{(1-\alpha)}} \geq \frac{1}{\sqrt{n}}
\]

2. Square both sides:
\[
\frac{E^2}{\sigma^2 Q_{(1-\alpha)}^2} \geq \frac{1}{n}
\]

3. Invert the inequality:
\[
n \geq \frac{\sigma^2 Q_{(1-\alpha)}^2}{E^2}
\]

- Use sample standard deviation \( s \) as an estimate for \( \sigma \) if population value is unknown.
- This formula will typically be provided on exam papers.

---

## Sample Size Estimation for Proportions

### Conservative Assumption

- Margin of error must match the form of proportion \( \pi \) (percentage or decimal).
- Standard error is maximized at \( \pi = 0.5 \), where:
  \[
  \pi(1 - \pi) = 0.25
  \]
- In percentages: \( \pi \times (100 - \pi) \leq 2500 \)

### Formula

\[
E \geq Q \times \sqrt{ \frac{2500}{n} }
\]

### Rearrangement

1. Divide both sides by \( Q \) and square:
\[
\left( \frac{E}{Q} \right)^2 \geq \frac{2500}{n}
\]

2. Invert inequality and multiply:
\[
n \geq \left( \frac{Q}{E} \right)^2 \times 2500
\]

---

## Worked Example: Mean

An IT training company wants to estimate the **average score** of self-study candidates:
- Population standard deviation: \( \sigma = 21.2 \)
- Desired margin of error: \( E = 3 \)
- Confidence level: 95% → \( Q = 1.96 \)

### Calculation

\[
n \geq \frac{(21.2)^2 \times (1.96)^2}{3^2} = 191.841
\]

**Required sample size: 192 students**

---

## Worked Example: Proportion

An IT journal wants to estimate the proportion of university students who own laptops:
- Confidence level: 95% → \( Q = 1.96 \)
- Margin of error: 2%

### Calculation

\[
n \geq \left( \frac{1.96}{2} \right)^2 \times 2500 = 2401
\]

**Required sample size: 2401 students**

---

# Sample Size Estimation (SSE)

## SSE for the Mean: Worked Example

### Problem
An IT training company wants to estimate the **average score** of self-study candidates.  
- Standard deviation \( \sigma = 21.2 \) points  
- Desired margin of error \( E = 3 \) points  
- Confidence level = 95% → \( Q_{(1-\alpha)} = 1.96 \)

### Formula
\[
n \geq \frac{\sigma^2 Q^2_{(1-\alpha)}}{E^2}
\]

### Calculation
\[
n \geq \frac{(21.2)^2 \times (1.96)^2}{3^2} = 191.841
\]

### Conclusion
The company must test **at least 192** candidates to achieve the desired confidence level.

---

## SSE for Proportions

We use the inequality:
\[
E \geq Q \times \sqrt{ \frac{\pi(1 - \pi)}{n} }
\]

### Conservative Assumption
- The standard error is maximized at \( \pi = 0.5 \)
- This gives \( \pi(1 - \pi) = 0.25 \)
- When using percentages: \( \pi \times (100 - \pi) = 2500 \)

So we simplify:
\[
E \geq Q \times \sqrt{ \frac{2500}{n} }
\]

### Rearranging to Solve for Sample Size
1. Divide both sides by \( Q \), then square:
\[
\left( \frac{E}{Q} \right)^2 \geq \frac{2500}{n}
\]
2. Invert and multiply:
\[
n \geq \left( \frac{Q}{E} \right)^2 \times 2500
\]

---

## SSE for Proportions: Worked Example

### Problem
An IT journal wants to estimate the proportion of university students who own laptops.  
- Confidence level = 95% → \( Q = 1.96 \)  
- Margin of error = 2%

### Calculation
\[
n \geq \left( \frac{1.96}{2} \right)^2 \times 2500 = 2401
\]

### Conclusion
The journal should survey **at least 2401** university students.

---
