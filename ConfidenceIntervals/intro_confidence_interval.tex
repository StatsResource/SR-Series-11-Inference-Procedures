Here’s your chapter on Confidence Intervals formatted and polished into clear, professional Markdown for readability and study:

---

# Chapter: Confidence Intervals

## Conceptual Overview

- The sample mean \( \bar{x} \) is an unbiased estimator of the population mean \( \mu \), but it's not exact.
- Mathematically, the probability of \( \bar{x} \) being equal to \( \mu \) is zero: \( P = 0 \).
- A **confidence interval (CI)** provides a range of values in which the population mean is likely to lie, based on sample data.

---

## Confidence Level

- The **level of confidence** (e.g., 95%) represents the long-run proportion of confidence intervals that will contain the true population parameter across repeated sampling.
- Common confidence levels:
  - 90% → \( \alpha = 0.10 \)
  - 95% → \( \alpha = 0.05 \)
  - 99% → \( \alpha = 0.01 \)

> Interpretation Note:  
> A 95% CI means that, over repeated studies, 95% of such intervals will contain the true value—not that there is a 95% probability the current interval contains the true value.

---

## Interval Construction

- Confidence intervals are typically centered around the point estimate (e.g. sample mean \( \bar{x} \)).
- The general form is:

\[
\text{Point Estimate} \pm (\text{Quantile} \times \text{Standard Error})
\]

### Components

- **Point Estimate**: Sample mean or proportion
- **Quantile**: Value from the normal or t-distribution corresponding to the desired confidence level
- **Standard Error**: Dispersion of the sampling distribution; decreases as sample size increases

---

## Distribution Criteria

- Use the **normal distribution** if:
  - Sample size is large (\( n > 30 \))
- Use the **t-distribution** if:
  - Sample size is small (\( n \leq 30 \))
  - Population is normally distributed
  - Population standard deviation \( \sigma \) is unknown

### Degrees of Freedom
\[
df = n - 1
\]

---

## Importance in Statistical Inference

Confidence intervals are foundational in making data-driven inferences:

- They provide an estimate with an associated measure of uncertainty.
- Widely used in hypothesis testing and decision-making.
- Can help assess whether observed differences are statistically meaningful.

> For example:  
> A drug increases heart rate by 2 bpm on average. If the 95% CI is (−5, 9), this suggests the true effect might be negative, and the observed increase could be due to chance.

---

## Common Misinterpretation

- It’s incorrect to say there is a 95% chance the parameter lies in the interval.
- That interpretation aligns with **Bayesian** credible intervals, not frequentist confidence intervals.

---
