

\section{Two Sample Mean - Inference Worked Example}
%%Question 3 Part b : Confidence interval for the difference in means of two samples.
%% b) 
The mean and standard deviation of the weights of a sample of Irish students according to sex are given below

\begin{center}
\begin{tabular}{|c|c|c|c|}\hline
& Number & Mean & Std. Dev. \\ \hline
Male &100&75&10 \\ \hline
Female&110& 66&8 \\  \hline

\end{tabular}
\end{center}


Calculate a 95\% confidence interval for the difference between the mean weight of all male
students and the mean weight of all female students.(7 males)

%%General Structure of a Confidence Interval



%------------------------------------------------------------- %
\noindent \textbf{Observed difference}

\begin{itemize}
\item let X denote the weights of male students    X= 75
\item let Y denote the weights of female students  Y= 66
\item The difference in the mean of weights X-Y= 9
\end{itemize}
%------------------------------------------------------------- %
\noindent \textbf{Quantile}

\begin{itemize}
\item Large sample (both groups are greater than 30).
\item Population variance is unknown.
\item Use t distribution with  degrees of Freedom.
\item Confidence level is 95\%. Therefore significance levels is 5%.
\item Confidence intervals are always two tailed procedures.
\item Column = k = $\frac{0.01}{2}= 0.005$
\end{itemize}

\begin{framed}
\noindent \textbf{Murdoch Barnes table 7}

\begin{itemize}
\item Row: df =  
\item Column = 0.005

\end{itemize}

Therefore the quantile is =  2.576 
\end{framed}

Stardard Error

%------------------------------------------------------------- %





Confidence Interval is therefore

\[99\% CI = 9 \pm (2.576 \times 1.256) \]


Part (ii)
Based on this confidence interval, test the hypothesis that on average male students are 6kgs heavier than female students.

State your hypotheses clearly. What is the significance level of this test?   (3 marks)

H0:X-y= 6




Since  we do not reject the null hypothesis at a significance level of 1\%.





Part (ii)
Based on this confidence interval, test the hypothesis that on average male students are 6kgs heavier than female students.
State your hypotheses clearly. What is the significance level of this test?   (3 marks)

\begin{description}

\item[Null :]                 True difference between means is zero

\item[Alternative:]         True difference between means is not zero
\end{description}
Or alternatively

\begin{description}
\item[Null :]              H0:X-Y= 0   True difference between means is zero

\item[Alternative:]         True difference between means is not zero
\end{description}



\begin{itemize}
\item Since the null value, 0 is inside the confidence interval  .

\item This means that the true difference of means could be zero.

\item We do not reject the null hypothesis at a significance level of 1\%.

\end{itemize}


%----------------------------------------------------------------------------------------------------------------------SLR%
\noindent \textbf{Test Statistic}

Remember the general structure of a test statistic

\begin{framed}
\[TS =  \frac{\mbox{Observed Value-Null Value}}{\mbox{Std. Error}} \]
\end{framed}


\begin{itemize}
\item Standard Error\[S.E.(D) = \frac{S_D}{n}=2.1378= 0.7555\]

\item Test statistis is a t random variable

\itemTest Statistic\[x-0S.E.(D)=2 - 00.755= 2.649\]
\end{itemize}

\noindent \textbf{Critical values}

\begin{itemize}
\item The sample size (n=8) is small ($n \leq 30$). Use t distribution with n-1 degrees of Freedom.
\item The test is one tailed.  k= 1  ( why?  ">" symbol in the alternative hypothesis).
\item Murdoch Barnes table 7
\item Column:  $\alpha/k = 0.05/1= 0.05$
\item Row: df = 7
\item Critical value =  1.895    
\end{itemize}
\noindent \textbf{Decision rule}

\begin{framed}
\noindent Is the absolute value of the test statistic value greater than the critical value?
\begin{itemize}
\item If Yes: we reject the null hypothesis

\item If No: We fail to reject the null hypothesis. (not enough evidence)
\end{itemize}
\end{framed}

Here TS = 2.64  is greater than CV = 1.895.

\noindent \textbf{Decision rule}
We reject the null hypothesis. Students do put on weight during college. 


#############################






