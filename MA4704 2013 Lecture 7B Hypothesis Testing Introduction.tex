
Here’s your collection of slides on hypothesis testing, cleaned and formatted into clear, structured Markdown for revision or presentation:

---

# Hypothesis Testing: Overview

- A **hypothesis test** is a method of making decisions based on experimental data.
- A result is said to be **statistically significant** if it is unlikely to have occurred by chance.

---

## Null and Alternative Hypotheses

- The **null hypothesis** \( H_0 \) is a statement about a population parameter, such as \( \mu \), the population mean.
- The goal is to test the viability of \( H_0 \) using sample data.
- The **alternative hypothesis** \( H_1 \) represents the opposite claim.
- Based on the data, we either **reject** or **fail to reject** \( H_0 \).

---

## Writing the Null Hypothesis

- Always refer to the **population parameter** (not the observed sample value).
- Must include a condition of **equality**:
  - \( = \), \( \leq \), or \( \geq \)

---

## One-Tailed vs. Two-Tailed Tests

- **Confidence intervals** are always **two-tailed**.
- **Hypothesis tests** may be one-tailed or two-tailed.
- The number of tails is determined by the **alternative hypothesis**:
  - If \( H_1 \) uses \( > \) or \( < \), → one-tailed
  - If \( H_1 \) uses \( \neq \), → two-tailed

---

## Significance Level \( \alpha \)

- Defines the **threshold** for rejecting \( H_0 \).
- Common choices:
  - \( \alpha = 0.05 \) → 5% level
  - \( \alpha = 0.01 \) → 1% level
- More conservative tests use smaller \( \alpha \) (e.g., 0.01).

---

## Making the Decision

- Compute the **observed difference**:  
  \[
  \text{Observed Statistic} - \text{Null Value}
  \]
- Find the **p-value**: probability of seeing a result as extreme or more extreme than the observed one, assuming \( H_0 \) is true.
- Compare to the significance level:
  - If \( p \leq \alpha \): reject \( H_0 \)
  - If \( p > \alpha \): fail to reject \( H_0 \)

---

## Interpreting Results

If a result is unlikely under \( H_0 \), we have two explanations:

1. A rare event occurred
2. \( H_0 \) is false

We typically infer that \( H_0 \) is false, **but it’s not proof**.

---

## Example: Die Throw

- A fair die should rarely yield a sum of 401.
- Simulation shows only 1.75% of such results occur.
- A biased die could explain the outcome.
- **Conclusion**: suggest bias, but don't prove it.
- Raises awareness of **Type I and II errors**.

---

## p-Value Clarification

- The p-value is **not** the probability that \( H_0 \) is true.
- Example: \( p = 0.0175 \) means:
  - Assuming \( H_0 \) is true, there is a 1.75% chance of seeing data this extreme.

---

## Critical Region and \( \alpha \)

- The **critical region** contains values that lead to rejection of \( H_0 \).
- The significance level \( \alpha \) is the probability of falling in this region **if \( H_0 \) is actually true**.

---
Here’s your material on hypothesis testing, test statistics, significance levels, and critical regions reformatted into clean, structured Markdown for clarity and study use:

---

# Hypothesis Testing: Key Concepts

## Significance Level (α)

- The **significance level** \( \alpha \) is the fixed probability of wrongly rejecting the null hypothesis \( H_0 \), if it is in fact true.
- It also represents the probability that the **test statistic** falls into the **critical region** when \( H_0 \) is true.
- Common choices: \( \alpha = 0.05 \) (5%) or \( \alpha = 0.01 \) (1%)

---

# Hypothesis Testing Procedures

We use two approaches for hypothesis testing:

### **Procedure 1: p-value Method**

1. State the null and alternative hypotheses.
2. Compute the test statistic.
3. Calculate the p-value.
4. Compare the p-value with \( \alpha \); reject \( H_0 \) if p-value is small enough.

### **Procedure 2: Critical Value Method** (used more often in exams)

1. State the null and alternative hypotheses.
2. Compute the test statistic.
3. Determine the **critical value** from the relevant distribution.
4. Apply the **decision rule** based on comparison with the critical value.

---

## Test Statistic (TS)

- A test statistic is derived from sample data and helps decide whether to reject \( H_0 \).
- It compares how far the observed value is from the expected value under \( H_0 \), relative to the variability.

### General Formula:

\[
TS = \frac{\text{Observed Value} - \text{Null Value}}{\text{Standard Error}}
\]

### Example:

- Observed value: 401  
- Null value: 350  
- Standard error: 17.07  

\[
TS = \frac{401 - 350}{17.07} = 2.99
\]

---

## Critical Value

- A **critical value** is a threshold used to determine whether the test statistic falls in the **critical region**.
- Depends on:
  - Significance level \( \alpha \)
  - Whether the test is **one-tailed** or **two-tailed**
- Can be found using Murdoch Barnes Table 7 or standard z/t tables.

---

## Critical Region

- The **critical region** is the range of values for the test statistic that leads to rejection of the null hypothesis.
- If the test statistic is beyond the critical value, it lies in the critical region → **reject \( H_0 \)**.

---

## One-Tailed Hypothesis Tests

- A **one-tailed test** rejects \( H_0 \) only if the test statistic lies in one extreme tail of the distribution.
- Usually applied when the alternative hypothesis is directional:
  - \( H_1: \mu > \mu_0 \) or \( H_1: \mu < \mu_0 \)
- Use \( k = 1 \) when computing quantiles from Murdoch Barnes Table 7.

---

Here’s your hypothesis testing revision material, reformatted into clear and structured Markdown for easier reading and study:

---

# Hypothesis Testing: Key Concepts

## Two-Tailed Hypothesis Test

- A **two-tailed test** rejects the null hypothesis \( H_0 \) if the test statistic falls into **either tail** of the distribution.
- Used when the **alternative hypothesis** contains \( \neq \)
- When consulting Murdoch Barnes Table 7, set \( k = 2 \)

---

## Critical Value

- A **critical value** is the threshold used to decide whether to reject \( H_0 \)
- Depends on:
  - Chosen **significance level** \( \alpha \)
  - **Tail type** (one-tailed or two-tailed)
  - **Sample size** → Use \( n - 1 \) degrees of freedom if small; otherwise use \( \infty \)
- For large samples at 5% significance in a two-tailed test, use \( 1.96 \) as critical value

---

## Critical Region and Decision Rule

- The **critical region** contains test statistic values that lead us to **reject \( H_0 \)**
- Compare the test statistic (TS) to the critical value (CV):

\[
\text{If } |\text{TS}| > \text{CV} \Rightarrow \text{Reject } H_0
\]
\[
\text{If } |\text{TS}| \leq \text{CV} \Rightarrow \text{Fail to reject } H_0
\]

### Example:
- Dice experiment:
  - TS = 2.99
  - CV = 1.96
  - Since \( |2.99| > 1.96 \), we reject the null hypothesis

---

## Hypothesis Testing Procedure: Summary

### Four Steps:

1. **State hypotheses**:
   - \( H_0 \): Null hypothesis  
   - \( H_1 \): Alternative hypothesis  
2. **Compute the test statistic**
3. **Determine the critical value**
4. **Apply decision rule**

---

## Hypothesis Test Conclusions

- We **never prove** the null hypothesis.
- We either:
  - **Reject** \( H_0 \)
  - **Fail to reject** \( H_0 \)

---

## p-values

- The **p-value** is the probability of obtaining a test statistic as extreme or more extreme than the observed one, assuming \( H_0 \) is true.
- If \( \text{p-value} < \alpha \), reject \( H_0 \)
- Used more commonly in software-based inference than the critical value method

---

## Interpreting p-values

- A **lower p-value** means stronger evidence against \( H_0 \)
- But p-values **do not** measure the probability that \( H_0 \) is true or false
- Reporting p-values allows the reader to interpret the result themselves, rather than simply stating significance

---





Here’s your notes cleaned and formatted as structured, readable Markdown for revision or presentation purposes:

---

# Paired t-Test: Summary

## Key Definitions

- \( \mu_d \): Mean of the population of case-wise differences  
- \( \bar{d} \): Mean of the sample of differences  
- \( s_d \): Standard deviation of the sample differences  
- \( n \): Number of paired observations  

---

## Test Procedure: Paired t-Test

To test the null hypothesis \( H_0: \mu_d = 0 \), follow these steps:

1. **Compute the case-wise differences**  
   \( d_i = y_i - x_i \)  
   Keep signs consistent.

2. **Calculate the mean difference**  
   \[
   \bar{d} = \frac{\sum d_i}{n}
   \]

3. **Compute the standard deviation of differences**  
   \[
   s_d = \sqrt{ \frac{\sum d_i^2 - n\bar{d}^2}{n - 1} }
   \]

4. **Find the standard error**  
   \[
   SE(\bar{d}) = \frac{s_d}{\sqrt{n}}
   \]

5. **Calculate the test statistic**  
   \[
   t = \frac{\bar{d}}{SE(\bar{d})}
   \]

- Degrees of freedom: \( df = n - 1 \)
- Use the t-distribution for inference.

---

## Hypothesis Summary

- **Null Hypothesis**: \( H_0: \mu_d = 0 \)  
- **Alternative Hypothesis**: \( H_1: \mu_d \neq 0 \)

---

# p-Values: Interpretation

- The **p-value** is the probability of obtaining a result as extreme or more extreme than the observed one, assuming \( H_0 \) is true.
- Compare the p-value with \( \alpha \) (e.g., 0.05):
  - If \( p < \alpha \), reject \( H_0 \)
  - If \( p \geq \alpha \), fail to reject \( H_0 \)
- Example: If \( p = 0.042 \) and \( \alpha = 0.05 \), reject \( H_0 \)
  - But if \( p = 0.0001 \), the decision is still to reject \( H_0 \)—though both suggest statistical significance, one is not “more valid” than the other

---

# Test Statistic for Known Variance

When testing a population mean and \( \sigma \) is known:

\[
Z = \frac{\bar{x} - \mu}{\sigma / \sqrt{n}}
\]

Use the Z-distribution rather than the t-distribution.

---

## Critical Region

- The **critical region** (or rejection region) contains values of the test statistic that lead us to reject the null hypothesis.
- Rule:
  - If \( |TS| > \text{Critical Value} \): **Reject \( H_0 \)**
  - If \( |TS| \leq \text{Critical Value} \): **Fail to reject \( H_0 \)**

---










